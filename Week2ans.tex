\documentclass[letter]{article}

%% Language and font encodings
\usepackage[english]{babel}
\usepackage[utf8x]{inputenc}
\usepackage[T1]{fontenc}
\usepackage{listings}
\usepackage{stmaryrd}

%% Sets page size and margins
\usepackage[top=2cm,bottom=2cm,left=3cm,right=3cm,marginparwidth=1.75cm]{geometry}

%% Useful packages
\usepackage{amsmath}
\usepackage{amsfonts}
\usepackage{amssymb}
\usepackage{amsthm}
\usepackage{graphicx}
\usepackage[colorinlistoftodos]{todonotes}
%\usepackage[colorlinks=true, allcolors=blue]{hyperref}
\usepackage{array}
\usepackage[shortlabels]{enumitem}
\usepackage[final]{pdfpages}
\usepackage[normalem]{ulem}
\usepackage{cancel}
\usepackage{xspace,mdwlist}
\usepackage{algorithmic}
\usepackage{mathtools}
\usetikzlibrary{calc}

\usepackage{courier} %% Sets font for listing as Courier.
\usepackage{listings, xcolor}
\lstset{
tabsize = 4, %% set tab space width
showstringspaces = false, %% prevent space marking in strings, string is defined as the text that is generally printed directly to the console
numbers = left, %% display line numbers on the left
commentstyle = \color{green}, %% set comment color
keywordstyle = \color{blue}, %% set keyword color
stringstyle = \color{red}, %% set string color
rulecolor = \color{black}, %% set frame color to avoid being affected by text color
basicstyle = \small \ttfamily , %% set listing font and size
breaklines = true, %% enable line breaking
numberstyle = \tiny,
}



\DeclarePairedDelimiter{\ceil}{\lceil}{\rceil}
\DeclarePairedDelimiter{\floor}{\lfloor}{\rfloor}

\newtheorem{theorem}{Theorem}[section]
\newtheorem*{claim}{Claim}
\DeclareMathOperator*{\argmin}{\arg\!\min}

\def\coursename{CS 201: Data Structures}

%% make title box
\newcommand{\header}[1]{%
	\begin{center}
		\fbox{
			\begin{minipage}{6in}
				\textbf{\coursename} \hfill       \\
				\textit{#1} \hfill \textit{\today}
			\end{minipage}
		}
	\end{center}
	\vspace*{4mm}
}

\def\problem#1#2#3{
\fbox{
\begin{minipage}{0.8\textwidth}
{\sc #1:}

\begin{description*}
\item[Given:] #2
\item[Find:] #3
\end{description*}
\end{minipage}
}
\bigskip
}


\begin{document}

\header{Week 2 Answer}

\begin{enumerate} [1.]
    \item 
    \begin{itemize}
        \item [(a)] True/false : method overriding is when we modify an inherited method, so that the method in the subclass has different behavior from that of the superclass. TRUE
        \item [(b)] 1b) True/false : method overloading is when we have more than one method with the same signature but different behavior. FALSE
        \item [(c)] 1c True/false : static methods can access instance variables, but can’t modify them. FALSE
        \item [(d)] 1d) Suppose Shape is an interface and Rectangle is a class implementing the interface Shape. What is the difference between x and y, in terms of what methods they can use?
        
        
        Shape x = new Rectangle();
        Rectangle y = new Rectangle();

        x can use implemented methods from Shape, but can’t use methods that are specific to Rectangle.
        y can use both the implemented methods from Shape and methods that are specific to Rectangle.
    \end{itemize}

    \item How does an interface differ from a class implementation? (Question from Carrano)

    Basically, an interface describes what a class that implements it does
    while the class implementation also includes how it's done. Interfaces don’t
    contain the contents of any methods, the exact way in which they carry out
    their goals.

    \item Write an interface for a Plant. Classes implementing Plant must be able to report their height and be watered (have their water level increase an integer amount).

    \begin{lstlisting}[language = Java , frame = trBL , firstnumber = 0 , escapeinside={(*@}{@*)}]
    /* An interface describing plant classes */
    public interface Plant {

        /* Returns the height of a plant 
        @return A int, the height */
        public int findHeight();

        /* Changes the water level by amount
        @param An integer, the amount the water level changes */
        public void water(int amount);
    }
    \end{lstlisting}

    \item Write a class Sunflower that implements Plant. 
    
    Instance variables:
    \begin{enumerate}
        \item height
        \item height
        \item numSeeds - the number of seeds in a Sunflower object.
    \end{enumerate}
    
    Methods:
    \begin{enumerate}
        \item All methods specified by the Plant interface
        \item howManySeeds() - returns numSeeds
    \end{enumerate}

    \newpage

    \begin{lstlisting}[language = Java , frame = trBL , firstnumber = 0 , escapeinside={(*@}{@*)}]
    class Sunflower implements Plant {
    private double height;
    private int waterLevel;
    private int numSeeds;
    
    public Sunflower(double height, int waterLevel, int numSeeds) {
        this.height = height;
        this.waterLevel = waterLevel;
        this.numSeeds = numSeeds;
    }
    public int findHeight() {
        return height;
    }
    public void water(int amount) {
        waterLevel += amount;
    }
    public int howManySeeds() {
        return numSeeds;
    }
    \end{lstlisting}
    
    \item Suppose I have a variable collection, which is a Plant[] array that stores different objects that implement the interface Plant. One of those objects is a Sunflower. Write a code to find which index of the Plant[] array contains a Sunflower object. If that index contains a Sunflower object, print out the index and the number of seeds.

    \begin{lstlisting}[language = Java , frame = trBL , firstnumber = 0 , escapeinside={(*@}{@*)}]
for (int i = 0; i < collection.length; i++) {
    if (collection[i] instanceof Sunflower) {
	int num = ((Sunflower)collection[i]).howManySeeds();
	System.out.println("index: " + i + " Number of seeds: " + num);
    }
}
    \end{lstlisting}

    
\end{enumerate}
\end{document}