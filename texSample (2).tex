\documentclass{article}
\usepackage{fullpage,amsmath,tabularx, listings}

\begin{document}

\noindent
Eric Alexander \\
CS 202, Winter 2021 \\
\textbf{\LaTeX template document}\\
\\
(This is a \emph{super} barebones file--I really encourage you to explore the fancier aspects of formatting over the course of the semester...)
\\
\\
\noindent
Check out this list:
\begin{enumerate}
	\item This is \#1 in the list!
	\item Number 2
		\begin{enumerate}
			\item You can even nest lists...
			\item ...like this!
		\end{enumerate}
	\item Number 3
	\begin{itemize}
		\item Or I can do bullets!
		\item Numbers are overrated, anyway...
	\end{itemize}

\end{enumerate}

\section{Sections!}

These are a good way to break up your Monday/Wednesday/Friday sections.

\section{Tables}
% By default, LaTeX will not indent the first paragraph, but will indent all later paragraphs that don't use \noindent.
% Also, this is a comment!
\noindent
Useful now are things like truth tables: \\

\begin{tabular}{c | c || c | c | c}

	P	& Q	& P $\land$ Q & P $\lor$ Q & P $\Rightarrow$ Q \\
	\hline
	T 	& T 	& T & T & T \\
	T 	& F 	& F & T & F \\
	F 	& T 	& F & T & T \\
	F 	& F 	& F & F & T \\

\end{tabular}

\section{Another section!}

See how the numbers are automatic? Isn't \LaTeX great?

\section{Math mode}

\texttt{mathmode} is how you make cool things like formulae and implication arrows and whatnot:

$$\forall x \in X, \quad \exists y \leq \epsilon$$
$$\forall y \in Y, \quad \exists x \leq \gamma$$
$${n! \over k!(n-k)!} = {n \choose k}$$

\texttt{mathmode} can be in a block, like above, or in-line. Consider an integer $a$ such that $a \in E$, or maybe think about how $x + y = z$...

% LaTeX will put in page breaks automatically, but if you want to force one, here's how
\newpage

\noindent
The \texttt{align} functionality is useful for structuring proofs, letting one easily follow from one line to the next:

\begin{align}
	(x+1)^2 &= (x+1)(x+1) \\
		&= x^2 + x + x + 1 \\
		&= x^2 + 2x + 1
\end{align}

\section{Code examples}

\lstset{language=Python, 
        basicstyle=\ttfamily\small, 
        showstringspaces=false,
	numbers=left,
	frame=single,
	tabsize=4,
	commentstyle=,
}
\begin{lstlisting}
# this is a comment

print('hello world')

for i in range(20):
	print('number {}'.format(i+1))
\end{lstlisting}

\section{Good luck!}

That should be enough to get you started. \textbf{Use the tools posted on Moodle!} They will be very helpful down the road.

\end{document}









