\documentclass[letter]{article}

%% Language and font encodings
\usepackage[english]{babel}
\usepackage[utf8x]{inputenc}
\usepackage[T1]{fontenc}
\usepackage{listings}

%% Sets page size and margins
\usepackage[top=2cm,bottom=2cm,left=3cm,right=3cm,marginparwidth=1.75cm]{geometry}

%% Useful packages
\usepackage{amsmath}
\usepackage{amsfonts}
\usepackage{amssymb}
\usepackage{amsthm}
\usepackage{graphicx}
\usepackage[colorinlistoftodos]{todonotes}
%\usepackage[colorlinks=true, allcolors=blue]{hyperref}
\usepackage{array}
\usepackage[shortlabels]{enumitem}
\usepackage[final]{pdfpages}
\usepackage[normalem]{ulem}
\usepackage{cancel}
\usepackage{xspace,mdwlist}
\usepackage{algorithmic}
\usepackage{mathtools}
\usetikzlibrary{calc}

\DeclarePairedDelimiter{\ceil}{\lceil}{\rceil}
\DeclarePairedDelimiter{\floor}{\lfloor}{\rfloor}

\newtheorem{theorem}{Theorem}[section]
\newtheorem*{claim}{Claim}
\DeclareMathOperator*{\argmin}{\arg\!\min}

\def\coursename{CS 201: Data Structures}

%% make title box
\newcommand{\header}[1]{%
	\begin{center}
		\fbox{
			\begin{minipage}{6in}
				\textbf{\coursename} \hfill       \\
				\textit{#1} \hfill \textit{\today}
			\end{minipage}
		}
	\end{center}
	\vspace*{4mm}
}

\def\problem#1#2#3{
\fbox{
\begin{minipage}{0.8\textwidth}
{\sc #1:}

\begin{description*}
\item[Given:] #2
\item[Find:] #3
\end{description*}
\end{minipage}
}
\bigskip
}


\begin{document}

\header{Week 1}

\begin{enumerate}[1.] 
    \item Answer the following questions (from WI20 prefect)
    \begin{itemize}
        \item [(a)] True / False: You need to recompile a Java program every time you make a code change and want to run it. 
        \item [(b)] True / False: By convention, the class name of a Java program usually matches the file name, but it doesn't have to.
        \item [(c)] True / False: You can get errors in Java both at compile time and at run time.
        \item [(d)] True / False: Java is a statically typed language, while Python is dynamically typed.
        \item [(e)] True / False: Primitive type variables are stored in the heap memory while reference type variables are stored in the environment.
        \item [(f)] What does the following code output, and why?
        \begin{verbatim}
        String x = "Hello";
        String y = x;
        x = "Hi";
        System.out.println(y);
        \end{verbatim}
    \end{itemize}

    
    \item What is stored in result after the following statements execute (Question from Koffman-Wolfgang)
    \begin{verbatim}
    StringBuilder result = new StringBuilder();
    String sentence = "Let’s all learn how to program in Java";
    String[] tokens = sentence.split(" ");
    for (String token : tokens) {
        result.append(token);
    }
    \end{verbatim}
        
    \item Look at the following code and answer the questions below

    \lstset{language=Java, 
        basicstyle=\ttfamily\small, 
        showstringspaces=false,
	numbers=left,
	frame=single,
	tabsize=4,
	commentstyle=,
}
    \begin{lstlisting}
public class Dog {
    private String breed;
    private int age;
    private static int dogCount = 0;

    public Dog(String breed, int age) {
        this.breed = breed;
        this.age = age;
        dogCount++;
    }

    public static int displayCount() {
        return dogCount;
    }

    public void setAge(int age) {
        if (age < 0) {
            System.out.println("Age must be >= 0");
        }
        else {
            this.age = age;
        }
    }

    public String Stats() {
        return "Breed = " + breed + " Age = " + Integer.toString(age);
    }       
}
    \end{lstlisting} Explain what the following keywords do:
    \begin{enumerate}[a)]
    	\item public (line 1)
            \item public (line 12)
    	\item private (line 2)
    	\item static (line 4)
            \item static (line 12)
            \item this (line 7)
            \item void (line 16)
    \end{enumerate}

    \item Write a \textit{CarletonStudents} class (in Java) with instance variables \textit{name}, \textit{major}, \textit{year}, \textbf{which can only be 2024, 2025, 2026, or 2027.} Include methods:
    \begin{itemize}
        \item \textit{setYear(int x)}, which sets the class year of the student. 
        \item \textit{getYear()}, which returns the student's class year
    \end{itemize}
    
     Think about what constructors you want and the security of the instance variables.
    

    
\end{enumerate}


\end{document}
